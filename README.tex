\documentclass{article}

\def\pin{p_\mathrm{in}}
\def\pout{p_\mathrm{out}}
\def\in{\mathrm{in}}
\def\out{\mathrm{out}}
\def\low{\mathrm{low}}
\def\high{\mathrm{high}}


\title{Dynamic community detection benchmark readme}

\begin{document}

\maketitle

This document describes the implementation of the dynamic network
community benchmark.


\section{Principles}
\label{sec:principles}

\begin{itemize}
\item The benchmark operates in intervals of time, not space.  That
  means that one cycle of the benchmark is time $\tau$ and snapshots
  are given at $0\tau$, $.01\tau$, etc, instead of one snapshot being
  the addition of one edge or one node.  Reason: For the merging
  benchmark, since $p$ is fixed, not the number of edges, there is a
  different number of edges in each benchmark.  It is ideal to have
  the critical points (completely split, completely merged) at the
  same time points in all benchmarks.  Further, when merging and
  expansion are coupled, we need a way to make a consistent time
  scale.  This is the easiest way.
\item The configurations are programmed as a certain state at a certain
  time, not as a dynamic process generating a certain state.  This
  means it easiest to always ensure that the model always matches a
  given ensemble.
\item Care is paid to time and space complexity of all operations.
  There are no $O(N^2)$ time or space operations.
\item The model is programmed with defensive programming techniques
  (e.g. arxiv:1210.0530) to prevent bugs, and will be fully tested.
\item Everything is fully modular.  A generic merging operation is
  defined between two communities, and this can be applied to any
  realization of the benchmark, possibly multiple times.  This leads
  to most flexibility and least possibility of bugs being introduced.
\item The benchmark is written in Python and depends only on the
  \texttt{networkx} package for graphs.  This makes for the most
  flexible deployment on the most number of systems and easiest to
  have flexibility and low time and space complexity.
\item The three ``standard'' benchmarks will be presets.  The defaults
  will be set as we want, but they can also have $n$ ,$q$, $\pin$,
  $\pout$, customized.  I can also have other ``advanced'' versions
  for testing.
\item Public and open source project.  git repository:
  https://git.becs.aalto.fi/rkdarst/dynbench
\end{itemize}


\section{Merging mechanism}

\begin{itemize}
\item Two communities, $A$ and $B$, are initialized with $n$ nodes.
\item Each edge within $A$ (between two nodes in $A$, excluding
  self-edges) is made with probability $\pin$.  These edges never
  change and remain static throughout the life of the benchmark.  The
  same is done for $B$.
\item The edge density of edges \textsl{between} communities ranges
  from $\pout$ in the split state to $\pin$ in the merged state.
  Thus, at the two endpoints, the graph is a standard SBM graph and at
  the other endpoint, an ER graph with $\pin$.  Note that the
  \textsl{actual} edge density (edges divided by edges possible) in
  the fully merged state can be lower than the internal densities of
  the two sides, but they are chosen from the same distribution.
\item The exact number of edges $m_\low$ in the split state is chosen
  from the binomial distribution $\mathcal{B}(n, \pout)$.  The exact
  number of edges $m_\high$ in the merged state is chosen from the
  binomial distribution $\mathcal{B}(n, \pin)$.  To choose any
  particular point, a linear interpolation is taken between these
  values.
  \begin{itemize}
  \item In particular, this means that at the halfway point, the
    number of edges is $(m_\high+m_\low)/2$, not chosen from a
    binomial distribution with $(\pin+\pout)/2$.  We are only in a
    random graph ensemble at the endpoints.
  \end{itemize}
\item Time is defined, with linear interpolation between these
  points.  Alternative formulations are easy to add due to the
  implementation above.
  \begin{itemize}
  \item $t'=0$, split state
  \item $t'={1 \over 2} \tau$, fully merged state.
  \item $t'=\tau$, split state.
  \end{itemize}
  $t'$ is the normalized time, $t' = t \mathrm{mod} \tau + \Phi$.  $\Phi$ is a
  phase factor.
\end{itemize}

\section{Expansion/contraction mechanism.}
\begin{itemize}
\item Two communities, $A$ and $B$, have $n$ nodes.  A fraction $f$
  defines the fraction of each community that is moved to the other
  community.  Our initial convention is $f=.5$.  Each community has an
  internal edge density $\pin$ and all links between any two nodes in
  different communities are present with independent probability
  $\pout$.
\item At the ``left'' side, $A$ has $n_A(1-f)$ nodes and $B$ has
  $n_b+f n_A$ nodes.  At the ``right'' side, $A$ has $n_A + f n_B$
  nodes and $B$ has $n_b(1-f)$ nodes.  (This can be generalized to
  different numbers of nodes in $A$ and $B$, as you see here.)
\item A time interpolation is done between
\item Time is defined, with linear interpolation between these
  points.  Alternative formulations are easy to add due to the
  implementation above.
  \begin{itemize}
  \item $t'=0$, equal sized communities.
  \item $t'={1 \over 4} \tau$, $A$ is smallest and $B$ is largest.
  \item $t'={1 \over 2} \tau$, equal sized communities.
  \item $t'={3 \over 4} \tau$, $A$ is largest and $B$ is smallest.
  \item $t'=\tau$, equal sized communities.
  \end{itemize}
  $t'$ is the normalized time, $t' = t \mathrm{mod} \tau + \Phi$.  $\Phi$ is a
  phase factor.
\item If a community is ever disconnected, raise an error and abort.
\end{itemize}

\section{Sample usage}

\begin{itemize}
\item \texttt{./bm.py StdMerge output-prefix}
\item \texttt{./bm.py StdGrow output-prefix}
\item \texttt{./bm.py StdMixed output-prefix}
\item \texttt{./bm.py StdMerge output-prefix --n=32 --q=16 --p\_in=.5 --p\_out=.1}
\item \texttt{./bm.py StdMerge output-prefix --k\_in=16 --k\_out=3}
\end{itemize}



\end{document}
